\usepackage{fontspec}
\usepackage[sfdefault, condensed]{roboto}  %% Fonte roboto
\usepackage{babel} % Idioma português
\usepackage{mathtools, amssymb, amsthm, makeidx, mathrsfs} % Pacotes matemáticos
\usepackage{booktabs} % Tabelas bonitas
    \renewcommand{\arraystretch}{1.2}
\usepackage[separate-uncertainty=true]{siunitx} % Pacote que lida com valores numéricos e unidades
	\sisetup{
		range-phrase=\text{\,a\,} ,	% \SIrange{1.5}{2.0}{m} como "1.0 m a 2.0 m"
		output-complex-root=\ensuremath{\mathrm{j}} , % j como unidade imaginária
		% exponent-product= ,
		% inter-unit-product=,
		output-decimal-marker={,},
		list-final-separator= \text{\,e\,},
		list-pair-separator = \text{\,e\,},
		} 
\usepackage{hyperref} % Hiperlinks (citações, referenciações e links externos)
\usepackage{graphicx} % Ambiente de figuras
\usepackage[dvipsnames, table, xcdraw]{xcolor} % Pacote de cores 
\hypersetup{ % Configurações para as cores dos hyperlinks
    colorlinks,
    linkcolor={blue!10!black},
    citecolor={blue!50!black},
    urlcolor={blue!80!black}
}
\usepackage{enumitem} % Pacote de ambientes de enumeração
\usepackage{physics} % Pacote com simplificações para ambientes de equação
\usepackage{subcaption}
\usepackage{cleveref} % Referenciar com nome do float
	\captionsetup[subfigure]{subrefformat=simple,labelformat=simple}
	\renewcommand\thesubfigure{(\alph{subfigure})}
\usepackage{setspace} % Espaçamento entre linhas
\setstretch{1.2}
\usepackage{indentfirst} % Começar texto com parágrafo
\usepackage{placeins} % Garantir que figuras (e outros floats) fiquem dentro da seção em que são chamadas
\usepackage{lipsum} % Gerador de texto para testar de formatação
\usepackage[superscript]{cite} % Citações
\usepackage{titlesec} % ESTILIZAÇÃO DE CAPÍTULOS
	\titleformat{\chapter}{\normalfont\LARGE\bfseries}{\thechapter}{1em}{}
	\titlespacing*{\chapter}{0pt}{3.5ex plus 1ex minus .2ex}{2.3ex plus .2ex}



% -------------------------------------------------------------------------------------------------------------------------------------------------------------
\newif\ifdraft % Modo draft para carregamento mais leve
% ----------------------------------------------------------------
% --------------------------- CABEÇALHOS --------------------------
\usepackage[firstpage=true,pages=some]{background}

\usepackage{geometry} % Rearrumar margens
	\geometry{%
		top=3cm,
		bottom=3cm,
		left=3cm,
		right=3cm,
		bindingoffset=0cm,
		headheight=30pt,
		headsep=1.5cm
	}
\usepackage{marginnote,calc} % Fazer faixa zenith ignorar margens
\usepackage{fancyhdr} % Cabeçalho customizável
% -----------------------------------------------------------------


% --------- primeira página -------------------------------------
% retirar tudo e colocar numeração à direita			  fancyhdr
\fancypagestyle{plain}{%
  \fancyhf{}
  \fancyfoot[R]{\thepage}
  \renewcommand{\headrulewidth}{0pt}
  \renewcommand{\footrulewidth}{0pt}%
}
% ----------------------------------------------------------------
\backgroundsetup{
			contents=,
			opacity=0,
			color=black,
		}

% ---------------------------------------------------------------
\newcommand{\setFaixa}{
%%	Faixa Zenith primeira página 			   			 background
	\ifdraft
		\backgroundsetup{
			contents=DRAFT,
			opacity=0.02,
			color=black,
		}
	\else
		\backgroundsetup{
			contents=\noindent\makebox[\textwidth]{\includegraphics[width=\paperwidth]{zenith-faixa.png}},
			scale=1,
			placement=top,
			opacity=1
		}
	\fi
%%	Logo zenith a direita						  		 fancyhdr
\pagestyle{fancy}
\fancyhf{}
\fancyhead[L,C,R]{}
\ifdefined\nomeNucleo
	\lhead{\small{\nomeNucleo\,--\,\Zenith}} % Núcleo
\else\ifdefined\nomeDepartamento
		\lhead{\small{\nomeDepartamento\,--\,\Zenith}} % Departamento
	\else
		\lhead{\small{\Zenith}} % Apenas Zenith
	\fi
\fi
\ifdraft
	\rhead{}
\else
	\rhead{\includegraphics[height=0.020\textheight]{LogoZ.jpg}}
\fi
\rfoot{\thepage}
\renewcommand\headrulewidth{0.5pt}
\renewcommand\footrulewidth{0pt}
}
% -----------------------------------------------------------------------

% -----------------------------------------------------------------------

\newcommand{\addFigure}[3] { %Parametros scale, fig_name, caption 
    \begin{figure}[!hbt]
      \centering
      \includegraphics[width=#1\linewidth]{figures/#2}
      \caption{#3}\label{fig:#2}
    \end{figure}
}

\newcommand{\addSubfigure}[3]{ % scale, fig_name, caption
	\centering
		\begin{subfigure}[htpb]{#1\textwidth}
			\centering
			\includegraphics[width=\linewidth]{figures/#2}
			\caption{#3}
			\label{fig:#2}
		\end{subfigure}
}
% --------------------------------------------------------------


% --------- FOLHA DE TÍTULO ------------------------------------
%  comandos
\newcommand{\nucleo}[1]{\def\nomeNucleo{#1}}
\newcommand{\departamento}[1]{\def\nomeDepartamento{#1}}
\newcommand{\protocolo}[1]{\def\numeroProtocolo{#1}}
\newcommand{\Zenith}{Zenith EESC USP}
\newcommand{\cidade}[1]{\def\nomeCidade{#1}}
\newcommand{\Resumo}[1]{\def\conteudoResumo{#1}}
\newcommand{\data}[1]{\def\hoje{#1}}
\newcommand{\Criacao}[1]{\def\dataCriacao{#1}}
\newcommand{\titulo}[1]{\def\nomeTitulo{#1}}
\newcommand{\tipo}[1]{\def\tipoDocumento{#1}}
\newcommand{\sumario}{\tableofcontents\thispagestyle{plain}\pagebreak} %fancyhdr
% --------------------------------------------------------------
%  titlepage
\newcommand{\geraTituloProtocolo}{
\clearpage
\begin{titlepage}
  \begin{center}
	\vspace*{2cm}
	\ifdefined\numeroProtocolo
	   \Huge \texttt{\textbf{\numeroProtocolo}} \\
	\fi
	\vspace*{0.5cm}
       \linespread{1}
       \LARGE \nomeTitulo \\
 \end{center}
	 \vfill

\ifdefined\conteudoResumo
	\begin{abstract}
		\conteudoResumo 
	\end{abstract}
	\vspace*{6cm}
 \fi
\centering
\ifdefined\hoje
	\hspace*{\fill} {\small Última atualização: \hoje} \\
\fi
\hspace*{\fill} {\small Data de criação: \dataCriacao}

\pagebreak
 \end{titlepage}
}

\newcommand{\geraTitulo}{
\ifdefined\nomeNucleo
	\ifdefined\nomeDepartamento
	    \author{\Large{\Zenith} \\ \nomeNucleo \\  \nomeDepartamento}
    \else
	    \author{\nomeNucleo\,--\,\Zenith}
	\fi
\else
    \ifdefined\nomeDepartamento
	\author{\nomeDepartamento\,--\,\Zenith}
    \else
	\author{\Large{\Zenith}}
	\fi
\fi
\title{\nomeTitulo}

\ifdefined\hoje     \date{\hoje}     \fi
\ifdefined\dataCriacao    \date{}   \fi   

\maketitle
\ifdefined\conteudoResumo
    \vfill
    \begin{abstract}
        \conteudoResumo
    \end{abstract}
    \ifdefined\dataCriacao % a página já será quebrada caso \dataCriacao esteja definido
    \else
        \vspace*{5cm}
        \thispagestyle{empty}\addtocounter{page}{-1}
        \pagebreak
    \fi
\fi

\ifdefined\dataCriacao
    \thispagestyle{empty}\addtocounter{page}{-1}
    \vfill
    \ifdefined\hoje
        \if\hoje\empty  % se for vazio, não fazer nada
        \else           % caso NÃO seja:
            \hspace*{\fill} Última atualização: \hoje \\
        \fi
    \else
        \hspace*{\fill} Última atualização: \today \\ % Escrever data caso linha seja apagada
    \fi
    \hspace*{\fill} Data de criação: \dataCriacao
    \pagebreak
\else
    \bigskip
\fi
}

\newcommand{\geraTituloProjeto}{
\begin{titlepage}
	\centering
	\vspace*{0.75cm}
	\begin{figure}[h]
	\centering
	\includegraphics[scale=0.1]{Logo.jpg}
	\end{figure}
	
	\vspace{0.75cm}
	{\Large \tipoDocumento \par}
	\vspace{1.5cm}
	{\LARGE \nomeTitulo \par}
	\vspace{2cm}
	{\large \nomeNucleo \par}
	\vspace{1cm}
	{\large \par}
	\vfill
	{\large \today \par}
	\vfill
	\end{titlepage}
	\newpage
}

\newcommand{\setTema}{
%%	Faixa Zenith primeira página 			   			 background
	\ifdraft
		\backgroundsetup{
			contents=DRAFT,
			opacity=0.02,
			color=black,
		}
	\else
		\backgroundsetup{
			contents=
			opacity=0
		}
	\fi
%%	Logo zenith a direita						  		 fancyhdr
\pagestyle{fancy}
\fancyhf{}
\fancyhead[L,C,R]{}
\ifdefined\nomeNucleo
	\lhead{\small{\nomeNucleo\,--\,\Zenith}} % Núcleo
\else\ifdefined\nomeDepartamento
		\lhead{\small{\nomeDepartamento\,--\,\Zenith}} % Departamento
	\else
		\lhead{\small{\Zenith}} % Apenas Zenith
	\fi
\fi
\ifdraft
	\rhead{}
\else
	\rhead{\includegraphics[height=0.020\textheight]{LogoZ.jpg}}
\fi
\rfoot{\thepage}
\renewcommand\headrulewidth{0.5pt}
\renewcommand\footrulewidth{0pt}
}
